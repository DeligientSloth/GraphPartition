% !TEX program = xelatex

\documentclass[UTF8,12pt]{ctexart} % 12pt 为字号大小
\usepackage{amssymb,amsfonts,amsmath,amsthm}
%\usepackage{fontspec,xltxtra,xunicode}
%\usepackage{times}
\renewcommand{\vec}[1]{\boldsymbol{#1}} % Uncomment for BOLD vectors.

%\usepackage{CJKutf8}

\usepackage{xeCJK}

\usepackage{indentfirst}
\setlength{\parindent}{2em}

\renewcommand{\baselinestretch}{1.4} % 1.4倍行距

%页边距
\usepackage[a4paper]{geometry}
\geometry{verbose,
    tmargin=2cm,% 上边距
    bmargin=2cm,% 下边距
    lmargin=2cm,% 左边距
    rmargin=2cm % 右边距
}

%图形相关
\usepackage[x11names]{xcolor} % must before tikz, x11names defines RoyalBlue3
\usepackage{graphicx}
\usepackage{pstricks,pst-plot,pst-eps}
\usepackage{subfig}
\def\pgfsysdriver{pgfsys-dvipdfmx.def} % put before tikz
\usepackage{tikz}

\usepackage[boxruled,algosection,linesnumbered]{algorithm2e}

\usepackage{listings}
\usepackage{color}

\usepackage{authblk}
\usepackage{abstract}

\definecolor{dkgreen}{rgb}{0,0.6,0}
\definecolor{gray}{rgb}{0.5,0.5,0.5}
\definecolor{mauve}{rgb}{0.58,0,0.82}

\lstset{
    frame=shadowbox,
    aboveskip=3mm,
    belowskip=3mm,
    showstringspaces=false,
    columns=flexible,
    basicstyle={\small\ttfamily},
    numbers=none,
    numberstyle=\tiny\color{gray},
    keywordstyle=\color{blue},
    commentstyle=\color{dkgreen},
    stringstyle=\color{mauve},
    breaklines=true,
    breakatwhitespace=true,
    tabsize=4,
    rulesepcolor=\color{red!20!green!20!blue!20},
}

%原文照排
\usepackage{verbatim}

\usepackage{url}

%习题环境
\theoremstyle{definition} 
\newtheorem{exs}{习题}

%解答环境
\ifx\proof\undefined\
\newenvironment{proof}[1][\protect\proofname]{\par
\normalfont\topsep6\p@\@plus6\p@\relax
\trivlist
\itemindent\parindent
\item[\hskip\labelsep
\scshape
#1]\ignorespaces
}{%
\endtrivlist\@endpefalse
}
\fi

\renewcommand{\proofname}{\it{解答}}

\begin{document}

\title{基于Spark的图划分算法实现}
\author[1]{李亮德}
\author[1]{王凌霄}
\author[2]{曹籽文}
\author[2]{侯诗铭}
\author[2]{路宏琳}
\author[3]{尉德利}
\affil[1]{中国科学院自动化研究所}
\affil[2]{中国科学院信息工程研究所}
\affil[3]{中国科学院大学人工智能学院}
\renewcommand\Authands{, }
% \date{} % 若不需要自动插入日期,则去掉前面的注释;{ } 中也可以自定义日期格式

\maketitle

\renewcommand{\abstractname}{}
\begin{onecolabstract}
    随着大数据时代的到来,数据的规模以前所未有的速度增长着,仅万维网的搜索引擎就可以抓取约一万亿的连接关系图。
    图作为一种由顶点和边构成的数据结构,能够简洁有力的表达事物之间的联系。
    对于一个大规模图的处理,必须进行图划分,降低分布式处理的各子图之间的耦合性,提高子图内部的连通性。
    最简单的随机划分算法利用哈希算法将数据划分到指定分区,
    谱方法将这一离散问题转换成计算矩阵的特征值以显示某种内在的连接关系,
    启发式算法通常将随机划分的结果作为初始划分,
    通过一些局部优化的方法减少交叉边的数量以达到降低通信代价的优化目的,
    层划分算法通过粗糙化、初始划分、细化三个阶段取得了较好的划分效果。
    Spark作为一个大数据平台,提供了全面、统一的框架用于管理各种有着不同性质的数据集和数据源。
    本文依托Spark平台,利用其提供强大的存储与计算能力,研究并实现了上述四种图划分算法,并对其划分性能进行了测试。
    结果表明:
\end{onecolabstract}

\newpage
\tableofcontents
\newpage

\section{研究背景}

随着大数据时代的到来,数据的规模以前所未有的速度增长着, Facebook、Twitter、微博等社交媒体每天都产生大量的社交图数据。
如何处理如此大规模的图数据成为目前研究的热点。
其中,图划分问题又是图数据处理领域中最为重要的问题之一,图的搜索,模式匹配等算法都需要图划分算法的支持。
本文依托近些年来兴起的大数据平台,利用其提供强大的存储与计算能力,研究并实现了以大数据处理平台Spark作为处理引擎的图划分算法。


图作为一种由顶点和边构成的数据结构,能够简洁有力的表达事物之间的联系。
在今天这个信息化社会中,随着互联网用户的急剧增加,越来越多的网络数据问题摆在了我们面前。
而在大数据时代下,大规模图数据处理问题便是一大热点。
类似于网状图,如果将每个网络用户看作图中的节点,而将用户与用户之间的关系看作图中的边,那么整个网络就可看作一张网络图,大量的网络图组成的集合便是图数据。

图划分是经典的组合优化问题,它的应用背景极其广泛,包括软硬件协同设计、VLSI 设计、并行计算中的任务分配等众多领域。
图划分是把一个图的顶点集分成$k$个不相交的子集,且满足子集之间的某些限制,即要找到一个图的划分,使得连接不同群组的边的权重尽可能小(意味着不同类中的点之间是不同的),而在群组内部的边有着很高的权重(意味着相同类中的点是彼此相似的)。
近年来,计算机技术正以难以想象的速度发展,继而各个领域的问题变得日益复杂(比如软硬件划分、大规模集成电路设计、任务分配等),因此,人们迫切地需要对图划分进行深入而广泛的研究。

考虑到大量图数据的实时性需求,我们采用了Spark框架进行实验。
Apache Spark是一个围绕速度、易用性和复杂分析构建的大数据处理框架。
Spark最初在2009年由加州大学伯克利分校的AMPLab开发,并于2010年成为Apache的开源项目之一。
与Hadoop和Storm等其他大数据和MapReduce技术相比,Spark有如下优势:

\begin{itemize}
    \item Spark提供了一个全面、统一的框架用于管理各种有着不同性质(文本数据、图表数据等)的数据集和数据源(批量数据或实时的流数据)的大数据处理的需求
    \item Spark可以将Hadoop集群中的应用在内存中的运行速度提升100倍,甚至能够将应用在磁盘上的运行速度提升10倍
\end{itemize}

本文综合前人的研究,首先介绍图划分算法的研究现状,重点介绍几类典型的图划分算法。
然后,通过对比实验,分析、比较不同图划分算法的性能特点。
最后,对图划分算法进行总结。

\section{图划分算法}

图数据划分问题是经典的NP(Non-deterministic Polynomial)完全问题,通常很难在有限的时间内找到图划分的最优解。
尽管其是难解问题,从20世纪90年代初期至今,国内外研究者不断对图划分及其相关问题进行深入研究,提出了许多性能较好的图划分算法。
现主要有随机划分法、谱方法、启发式方法、多层划分算法等。

\subsection{随机划分}

散列划分是最经典的随机图划分算法之一。
每个节点都有唯一的ID,散列划分通过一个Hash函数计算各ID的哈希值,然后将哈希值相同的节点分配到相同的分区中。
散列划分的优势在于简单易实现,不需要额外的开销且划分均衡可控。
但这种方法的缺陷也是非常明显的,它没有考虑到图的内部结构,各节点会被随机地划分到各分区中,结果就是区间边会非常多,对后续的操作造成很大不便,并产生巨大的通信开销。

\subsection{谱方法}

1990年,R.Leland和B. Hendrickson提出了谱方法[1]。
谱方法主要是针对图的二划分而言的。
“谱”主要指Laplacian矩阵的应用,其具体物理意义应该是反映了图中各种节点之间的某种内在的连接信息。

谱方法是将图划分这一离散问题转换成计算矩阵的Fiedler特征值的方式。
具体而言,就是先将图的表示从邻接矩阵$A$转换为对应的拉普拉斯矩阵$L=A-D$,
其中$D$为一个对角阵,对角元素$D_{ii}=deg⁡(i)$;
$L$矩阵中的元素定义如下所示:

\begin{equation}
    L_{i j}=\left\{\begin{array}{cc}{1,} & {\text { if } e(i, j) \in E} \\ {-\operatorname{deg}(i),} & {\text { if } i=j} \\ {0,} & {\text { other }}\end{array}\right.
\end{equation}

基于图谱理论的划分方法的主要思想是:
如果图$G$是连通的,那么其第二小特征值(最小的是$0$)也就是Fiedler特征值将给出图的连通性的一个度量。
如果将每个节点按对应的Fiedler特征值进行排序,然后将排好的序列分成$k$个部分,每个部分的节点将构成一个分区,即可实现图划分。
然而从计算量上来看,Fiedler特征值的计算和排序都较为复杂。
谱方法能为许多不同类的问题提供较好的划分,但是谱方法的计算量非常大。

\subsection{启发式算法}

启发式算法通常是在对图已经进行过划分的结果上对其进行优化调整的。
启发式算法的划分结果很大程度的依赖初始划分的结果,为获得较好划分结果,需要与能够产生好较优划分的算法结合。
为了提高算法的执行效率,启发式算法通常将随机划分的结果作为初始划分,然后通过一些局部优化的方法减少交叉边的数量以达到降低通信代价的优化目的。

W.Kernighan和S. Lin 提出了Kernighan-Lin算法[2],它是一种比较典型的基于启发式规则的求解策略。
KL算法是一种贪婪算法,其初始划分选取了递归二分法。
首先将图随机划分成两个子图且要求子图规模尽量一致即保证负载均衡,接着随机交换划分后两个子图的任意两个顶点并对其进行标记,依次选取未被标记的顶点进行交换,当所有标记过的顶点都经过交换之后,结束一轮迭代过程。
记录每次迭代的过程中记录的交叉边个数最少,即划分质量最优的时顶点的划分状态,并将这种状态作为下次迭代的初始化分,以此类推。
当下次迭代对划分质量无明显改善时,停止迭代。
KL算法流程图如下图所示。

KL算法的时间复杂度为O(t*n*n) (n为节点数,t为迭代次数)。
所以当图的数据很大时,执行时间过长,因此不适用于大规模图。
此外,KL算法是初始解敏感的算法,即产生一个较好初始社区的条件是事先得知社区的个数或平均规模,若初始解较差,则会造成收敛速度缓慢,最终解较差。
由于现实世界中的社区无法事先预知,故KL算法的实用价值不大。

\subsection{多层划分算法}

为了处理规模较大的图,文献[3]提出了多层的图划分框架METIS。
多层划分算法包括三个阶段:粗糙化 (coarsening) 阶段、初始划分、细化阶段(uncoarsening)。
第一阶段通过粗糙化技术将大图$G=(V,E)$约化为可接受的小图;
第二阶段将第一阶段获得的小图进行随机划分,并进行优化;
第三阶段通过细化技术以及优化技术将小图的划分还原为原图的划分。
该算法广泛地应用在各类大图的划分,对于百万规模以内的图,通常具有较好的实际效果。

\subsubsection{粗化阶段}

在粗化阶段,一些联系较强的部分节点会聚集形成一个局部整体,这个局部整体会以一个带节点权重的超级节点 (Super node) 的形式呈现出来,而构成该局部整体的节点以及它们之间的边会暂时的从图中隐藏掉,同时对外只呈现出一个权重节点。为了得到这些局部整体,需要将部分图顶点进行融合。顶点融合的最终目的是为了减小原始图的规模,因此顶点的匹配应该最大化。最大匹配的定义如下:如果一个匹配在不使两条边指向同一个顶点的情况下无法再增加新的边,这个匹配叫做图的最大匹配。(由于匹配的计算方法不同,最大匹配的结果可能会有差异)。

经常用到的寻找匹配的策略主要有两种:随机策略 (Random Matching,RM ) 和权重边策略 (Heavy edge matching,HEM)。

\paragraph{随机策略}
随机策略算法顾名思义,即随机选取边来组成一个匹配,其时间复杂度为O(m)。
用随机算法可以快速有效地生成一个匹配随即最大匹配算法,按照如下步骤工作:
先按照随机的顺序遍历原始图中所有的顶点,如果一个顶点u还没有被匹配,算法会随机地选择其尚未匹配的相邻顶点。
如果存在这样的相邻顶点v,就将边(u,v)加入匹配中。
如果己经不存在未配的相邻顶点,则顶点u标记为未匹配顶点。
这种策略虽然可以有效地找出一个极大匹配,但却没有考虑边和节点的权重值等信息。

\paragraph{权重边策略}
权重边匹配即在用重边策略寻找匹配时要尽可能地选择边权重值最大的边,是一种简单有效的求取最大匹配的方法。
在粗化过程中生成的各图的节点数在不断减少,但各图中节点的权重之和始终保持不变,但各图中的边权重之和却在持续减小。
对于粗化过程中生成的两个连续的图$G_i=(V_i,E_i)$和$G_(i+!)=(V_(i+1),E_(i+1))$以及从$G_i$生成$G_(i+1)$过程中的一个匹配$M_i \in E_i$。
如果A是边的集合,定义$W(A)$为$A$中边的权值的总和。
显然可以得到:$W(E_(i+1) )=W(E_i )-W(M_i )$
其中,$W(E_i)$和$W(E_(i+1))$分别表示图$G_i$和$G_(i+1)$中的边的权重值之和,而$W(M_i)$则表示匹配$M_i$中的所有边的权重值之和。

由此可见,粗化后的图中的边权重之和与粗化过程中的匹配有很大的关系。如果匹配中的边权重之和越大,则粗化后的图中边权重之和越小,反之亦然。所以,HEM使得匹配中的边权重之和最大化,即在每一次的粗化过程中尽可能多的减少边权重,从而在粗化过程结束时得到的图中的边权重之和最小。

\subsubsection{初始划分阶段}

Metis第二步是对粗化后的图进行k路划分。
对粗化图$G_m=(V_m,E_m)$计算划分$P_m$使得划分后的每部分大致均勾地含有原图的$|V|/k$个顶点。

一种产生k路初始划分的办法就是不断地对原图进行粗化操作,直到粗化图只剩下k个顶点。
这个含有k个顶点的粗化图可以作为原始图的k路初始划分。
在这个过程中会产生两个问题:
\begin{enumerate}
    \item 对于很多图来说,在进行了几次粗化之后,每一次的粗化过程所能减少的图的规模过小,因此粗化耗费的资源会很大。
    \item 即使将原图粗化到了仅剩个顶点,这些顶点的权值也极有可能差异很大,最终导致初始划分的平衡度大大降低。
\end{enumerate}

多级划分算法是算法的基本思想,就是在进行了顶点融合算法之后,原图的规模已经减小到了比较容易处理的地步。
经过顶点粗化的处理以后,顶点数和边数大大减少的新图将更有利于初始划分的进行。
在多级划分中,初始划分就是现将图进行二路划分,并在划分的同时考虑负载的平衡。
初始划分常用的算法是几何划分和谱划分等。
由于概化图规模一般较小,执行以上算法通常耗时很少。

\subsubsection{细化阶段}

如前所述,细化阶段是粗化阶段的反过程,其实就是还原过程。
在细化过程中,在粗化过程中被隐藏的边和节点将会逐步重新呈现出来,

在这个步骤中,粗化图的划分,会通过回溯每一级的粗化图$G_m$的划分$P_m$还原成原图。
虽然$P_(i+1)$是$G_(i+1)$的局部最小划分,但是细化后的划分$P_i$可能不再是$G_i$的局部最小划分。
由于$G_i$更加精细,所以会有更大的自由度优化$P_i$以减少边割。
因此,仍然可以使用局部细化启发式算法例如KL算法来优化划分$G_{i+1}$的划分。
每进行一次细化,算法会对细化后的划分使用优化算法。
划分优化算法的最基本思想是在划分后的两个部分中选择两个顶点集合进行互换,如果得到的新划分有更小的边割,则采用新的划分。

\subsubsection{整体算法流程}

\begin{algorithm}[H]
\SetAlgoLined
\KwIn{a differentiable action-value function parameterization:
        $\hat{q} : \mathcal{S} \times \mathcal{A} \times \mathbb{R}^{d} \rightarrow \mathbb{R}$}
\KwIn{a policy $\pi$ (if estimating $q_{\pi}$ )}
Algorithm parameters: step size $\alpha \in (0,1]$, small $\epsilon > 0 $, a positive integer $n$\;
Initialize value-function weights $\mathbf{w} \in \mathbb{R}^{d}$ arbitrarily (e.g., $\mathbb{w} = 0$)
All store and access operations (for $S_t$, $A_t$, and $R_t$) can take their index mod $n + 1$ \;

\For{each episode}{
    Initialize and store $S_0 \neq terminal$ \;
    Select and store an action $A_0 \sim \pi(\cdot | S_0)$ 
    or $\epsilon-greedy$ wrt $\hat{q}\left(S_{0}, \cdot, \mathbf{w}\right)$\;
    $T \leftarrow \infty$ \;
    \For{$t = 0,1,2,\cdots$}{
        \uIf{$t < T$}{
            Take action $A_t$ \;
            Observe and store the next reward as $R_{t+1}$ and the next state as $S_{t+1}$ \;
            \uIf{$S_{t+1} = terminal$}{
                $T \leftarrow t + 1$ \;
            }
            \uElse{Select and store an action $A_{t+1} \sim \pi(\cdot|S_{t+1})$
            or $\epsilon-greedy$ wrt $\hat{q}\left(S_{t+1}, \cdot, \mathbf{w}\right)$\;
            }
        }
        
        $\tau \leftarrow t-n+1$ ($\tau$ is the time whose estimate is being updated) \;
        \uIf{$\tau \geq 0$}{
            $ G \leftarrow \sum_{i=\tau+1}^{\min (\tau+n, T)} \gamma^{i-\tau-1} R_{i} $ \;
            \uIf{$\tau + n < T$}{
                $G \leftarrow G+\gamma^{n} \hat{q}\left(S_{\tau+n}, A_{\tau+n}, \mathbf{w}\right)$
            }

            $\mathbf{w} \leftarrow 
            \mathbf{w}+
            \alpha\left[G-\hat{q}\left(S_{\tau}, A_{\tau}, \mathbf{w}\right)\right] 
            \nabla \hat{q}\left(S_{\tau}, A_{\tau}, \mathbf{w}\right)$
        }
    } % End of step loop
    Until $\tau = T - 1$

} % End of episode loop
\caption{Episodic semi-gradient $n$-step Sarsa for estimating $\hat{q} \approx q_* $ or $q_\pi$}
\end{algorithm}

\section{基于Spark的图划分}

Apache Spark是一个开源集群运算框架,最初是由加州大学柏克莱分校AMPLab所开发。
相对于Hadoop的MapReduce会在运行完工作后将中介数据存放到磁盘中,Spark使用了存储器内运算技术,能在数据尚未写入硬盘时即在存储器内分析运算。
Spark在存储器内运行程序的运算速度能做到比Hadoop MapReduce的运算速度快上100倍,即便是运行程序于硬盘时,Spark也能快上10倍速度。

Apache Spark项目包含下列几项:弹性分布式数据集(RDDs)、Spark SQL、Spark Streaming、MLlib和GraphX。
Spark提供了分布式任务调度,调度和基本的I/O功能。
Spark的基础程序抽象是弹性分布式数据集(RDDs),RDD一个可以并行操作、有容错机制的数据集合。
RDDs可以透过引用外部存储系统的数据集创建(例如:共享文件系统、HDFS、HBase或其他 Hadoop 数据格式的数据源),
或者是透过在现有RDDs的转换而创建(比如:map、filter、reduce、join等等)。 

\subsection{图数据实现}

为了在Spark上进行图划分,本文将先介绍\texttt{Node}和\texttt{Graph}这两个类的基本情况。
\texttt{Graph}类主要用\texttt{edgeRDD}和\texttt{nodeRDD}表示以及进行分布式存储和计算。

\begin{itemize}
    \item \texttt{edgeRDD}的每一条记录用\texttt{sourceNode,targetNode,weight,isMatch}表示。
    这四个部分组成的意义分别是:起点节点Id值、终点节点Id值、边权重、是否已经被匹配。
    
    \item \texttt{nodeRDD}的每一条记录是一个\texttt{Node}类,\texttt{Node}类的属性主要包括:
    idxneighbour、E、I、partition、chosen(是否被KL算法交换过)、由哪些点组成(用于Metis算法中)、isMark(是否是匹配点)、weight(用于Metis组合点)。

\end{itemize}

\texttt{Graph}类和\texttt{Node}类包含了大量的基础函数,包括
\begin{itemize}
    \item 根据输入的\texttt{edge},构建\texttt{nodeRDD}
    \item 根据分区结果计算\texttt{Node}的属性,包括$E$和$I$
    \item 交换了两个点之后,应该如何调整$E$和$I$
    \item 图划分指标的计算
    \item $\cdots$
\end{itemize}

\subsubsection{\texttt{Node}类}

\texttt{Node}类是图数据处理的基础。

\begin{table}[htbp]
    \centering
    \caption{\texttt{Node}类主要属性}
    \begin{tabular}{ccc}
        \hline
        属性& 类型 & 定义\\
        \hline
        idx         & String             & 节点的唯一Id \\
        neighbour   & Map[String,Double] & 节点的所有近邻节点\\
        E           & Double             & 外部权重,即节点与其他子图内的节点的连接权重和\\
        I           & Double             & 内部权重,即节点与本子图内的节点的连接权重和\\
        partition   & Int                & 节点所在子图的Id\\
        chosen      & Boolean            & 节点是否与其他子图的节点交换过\\
        composition & List[Node]         & 该节点的组成节点列表(仅用于节点聚合/拆分过程)\\
        composLevel & Int                & 该节点的聚合程度(仅用于节点聚合/拆分过程)\\
        isMark      & Boolean            & 节点是否与其他节点匹配过 \\
        weight      & Double             & 节点的权重\\
        \hline
        \centering
    \end{tabular}
\end{table}

\subsubsection{\texttt{Graph}类}

\texttt{Graph}类是将节点和边组合起来。

\begin{table}[htbp]
    \centering
    \caption{\texttt{Graph}类主要属性}
    \begin{tabular}{ccc}
        \hline
        属性& 类型 & 定义\\
        \hline
        nodeNum & Long & 图内节点的个数 \\
        edgeRDD & RDD[(Str, Str, Double, Bool)] & 图内所有边数据的RDD形式\\
        nodeRDD & RDD[Node] & 图内所有节点的RDD形式\\
        \hline
        \centering
    \end{tabular}
\end{table}


\subsection{哈希划分算法实现}

通过map算子将\texttt{graph}里面的\texttt{nodeRDD}转化为(idx,分区号)键值类型的RDD。

根据输入的划分数$k$,对\texttt{Node}的Id重新进行哈希划分,具体公式是$hash(idx)\%k$。

通过在Spark executor中的taskcontext得到每一条记录的分区号,得到(idx,分区号)的键值对RDD形式。
然后调用\texttt{graph.buildPartitionGraph}方法,根据得到的分区信息更新\texttt{graph}的\texttt{nodeRDD}信息。

\begin{lstlisting}[language=Scala]
import org.apache.spark.{HashPartitioner, TaskContext}
import util.Graph

object HashGraphPartition {
    def partition(graph: Graph, partitions: Int): Graph = {
        val assigenment = graph.nodeRDD.map(x => (x.getIdx, 0)).partitionBy(
            new HashPartitioner(partitions)).map(x => (x._1, TaskContext.getPartitionId))
        graph.buildPartitionGraph(assigenment)
    }
}
\end{lstlisting}

\subsection{谱聚类算法实现}

谱聚类的核心思想是最小化子图之间的连边权重。
具体做法是对拉普拉斯矩阵求最小的$k$个特征值对应特征向量。
为了控制子图的规模,需要用度矩阵(相似度矩阵的行和)进行归一化。
这里有两个挑战:
\begin{enumerate}
    \item 对拉普拉斯矩阵进行特征值分解的复杂度非常高($O(N^3)$),而且很难实现并行化。
    \item 在Metis算法中,如果每个\texttt{node}的权重不一样,很可能产生不平衡分割。
\end{enumerate}

第一个挑战的解决方案可以参考Frank Lin和William W.Cohen发表于ICML 2010的论文,他们提出的幂迭代算法将特征值分解转化为矩阵的迭代乘积。

因为矩阵乘积的并行化在Spark底层已经实现并且优化,所以并行化更加简单。
具体来说,就是在数据归一化的逐对相似矩阵上,使用截断的幂迭代,寻找数据集的一个超低维嵌入。
这种嵌入恰好是很有效的聚类指标,使它在真实数据集上总是好于广泛使用的谱聚类方法。
算法的伪代码如下:

\begin{algorithm}[htbp]
\caption{PIC算法流程}
\SetAlgoLined
\KwIn{按行归一化的关联矩阵$W$}
\KwIn{期望聚类数$k$}
随机选取一个非零初始向量$v^0$ \\
\Repeat{$\left|\delta^{t}-\delta^{t-1}\right| \simeq 0$}{
    $ v^{(t+1)}=\frac{W v^{(t)}}{\left|W v^{(t)}\right|_{1}} $\\
    $\delta^{(t+1)}=\left|v^{(t+1)}-v^{(t)}\right|$
    增加$t$值
}
使用k-means算法对向量$v^t$中的点进行聚类
\KwOut{类$C_1,C_2,\cdots,C_K$}
\end{algorithm}

\subsection{Kernighan-Lin算法实现}

Kernighan-Lin算法的思想迭代贪心优化。
对于给定的图划分,Kernighan-Lin算法尝试交换不同分区的点,计算交换带来的增益,选择增益大于$0$并且最大的点来交换,所有点对交换的增益都小于等于$0$。
对于Spark分布式计算而言,KL算法首先会根据输入的初始化图划分计算,构建划分图。

本文提供了好几种构建划分图的模式,这里介绍两种——
\begin{enumerate}
    \item 基于\texttt{nodeRDD}的方法,也就是针对\texttt{nodeRDD}已经存在的情况构建
    \item 基于\texttt{edgeRDD}的方法,也就是\texttt{nodeRDD}还不存在的情况
\end{enumerate}

基于\texttt{nodeRDD}的核心是根据邻居是否与自己在一个分区来计算这个节点跟内部和外部的连接权重。

基于\texttt{nodeRDD}的划分方法的瓶颈在于在数据量很大的时候,基于map表查询的方法存在性能瓶颈。
更好的方法是通过join来实现节点邻居和分区信息的聚合,具体做法如下:

\begin{lstlisting}[language=Scala]
def buildPartitionGraph():Graph={
    if(this.nodeRDD==null) buildGraph()
    val map_idx_partition = this.nodeRDD.map(x=>(x.getIdx,x.getPartition)).collectAsMap()
    this.nodeRDD=this.nodeRDD.map(
        x=>{
            val neighbour = x.getNeighbour
            var E = 0.0
            var I = 0.0
            for (elem <- neighbour) {
                if(map_idx_partition.contains(elem._1)){
                    if(map_idx_partition(elem._1)==x.getPartition)
                        I+=elem._2
                    else
                        E+=elem._2
                }
            }
            x.setE(E).setI(I)
        }
    )
    this
}
\end{lstlisting}

随后KL算法会更新\texttt{nodeRDD}里面的各个属性,特别是$E$和$I$,$E$和$I$在KL算法执行过程中会迭代更新。

在迭代的每一步,KL算法首先是确定最大增益的点。
它的优点是足够准确,需要的迭代次数较少,缺点是每次需要计算大量点对的增益。
在Spark分布式实现上,具体做法是求出\texttt{nodeRDD}的笛卡尔积,去除一些没必要计算的点之后求出点对之间的增益。
只保留大于$0$的增益,然后通过reduce算子求出增大增益的点对。

\begin{lstlisting}[language=Scala]
def getMaxGain(nodeUnChosen: RDD[Node]): (Node, Node, Double) = {
    val node_gain = nodeUnChosen.cartesian(nodeUnChosen).filter(
        x => x._1.getPartition != x._2.getPartition
    ).map(x => {
        (x._1, x._2, x._1.swapGain(x._2))
    }).persist()

    val pos_gain = node_gain.filter(_._3 > 0)
    if (pos_gain.isEmpty()) null
    else pos_gain.reduce((x, y) => {
        if (x._3 >= y._3) x else y
    })
}
\end{lstlisting}

针对最大增益需要计算大量点对的情况,本项目提出了两种优化思路:

\begin{enumerate}
    \item Stochastic Max Gain
    每次只找到增益大于$0$的点就交换,类似于机器学习里面的随机梯度下降,整个优化过程非常曲折。
    这种策略每次交换的增益很低,比较难收敛。

    \item Mini-Partition Max Gain
    可以把\texttt{nodeRDD}或者\texttt{nodeRDD}的笛卡尔积分成很多个分区,每次随机取一个分区计算最大增益的点对进行交换。
    这个策略借用了机器学习里面对Mini Batch的数据进行Gradient Descent的精神,不过暂时没有实现。
\end{enumerate}

选择完交换的点之后,需要更新相关的点的$E$和$I$。
具体做法是调用\texttt{graph}类的\texttt{swapUpdate}方法,内部通过用map算子。
对\texttt{nodeRDD}每一个节点记录分布式调用\texttt{Node}的\texttt{swapUpdate}方法,内部通过用map算子。
\texttt{Node}的\texttt{swapUpdate}方法会根据点与交换两个点的连接关系更新$E$和$I$,每一次交换,程序都会记录性能的变化,便于观察。
% 伪代码如下

% \begin{lstlisting}[language=Scala]
% graph.swapUpdate的细节是调用点的swapUpdate方法:
% this.nodeRDD = this.nodeRDD.map(
%             x => x.swapUpdate(swap_node_a, swap_node_b)
%         )
% \end{lstlisting}

% input:node d
% out put:update node d
% if 节点d是节点a本身
%     d.E = a.I+edgeWeight(a,b)
%     d.I = a.E-edgeWeight(a,b)
% else if 节点d是节点b本身
%     d.E = b.I+edgeWeight(a,b)
%     d.I = b.E-edgeWeight(a,b)
% else if 节点d跟节点a在一个分区
%     d.E = d.I-edgeWeight(d,a)+edgeWeight(d,b)
%     d.I = d.E+edgeWeight(d,a)-edgeWeight(d,b)
% else if 节点d跟节点b在一个分区
%     d.E = d.I+edgeWeight(d,a)-edgeWeight(d,b)
%     d.I = d.E-edgeWeight(d,a)+edgeWeight(d,b)

对应的scala代码为:

\begin{lstlisting}[language=Scala]
if (is_node_a)
    return this.setE(I_a + weight_ab).setI(E_a - weight_ab).
            setChosen(true).setPartition(swap_node_b.getPartition)
if (is_node_b)
    return this.setE(I_b + weight_ab).setI(E_b - weight_ab).
            setChosen(true).setPartition(swap_node_a.getPartition)

val weight_a = this.edgeWeight(swap_node_a)
val weight_b = this.edgeWeight(swap_node_b)

if (weight_a == 0.0 && weight_b == 0.0) return this

if (in_a_graph)
    this.setI(
        this.getI - weight_a + weight_b
    ).setE(
        //这些点的E增加了与a连接的权重
        this.getE + weight_a - weight_b
    )
else
    this.setI(
        this.getI + weight_a - weight_b
    ).setE(
        //这些点的E增加了与a连接的权重
        this.getE - weight_a + weight_b
    )
\end{lstlisting}

\subsection{Metis算法实现}

Metis算法主要是通过最大匹配,合并匹配点,来实现图的粗化。
得到粗化的图之后调用一些图划分算法(比如谱聚类),将图划分为k个子图。
最后对图进行细化,也就是将原来粗化的图一步步还原回来。
由于在粗化的图上面进行图划分会有信息丢失,还原之后得到图划分不一定是最优的图划分。
因此Metis算法需要用Kernighan-Lin算法实现微调。

\subsubsection{粗化}

Metis的粗化过程会一直执行,直到图中的点小于$c*k$个,其中$k$是程序输入,表示划分成几个图,$c$是超参数,一般取$10~15$。

\begin{lstlisting}
while(graph.nodeNum<c*k)
   找出图中所有的最大匹配边
   合并最大匹配的边对应的两个点
   更新节点和边的状态
\end{lstlisting}

其中最大匹配的伪代码如下:

\begin{lstlisting}
while(还存在可以匹配的边)
    找出下一条匹配的边
    合并两个点,更新nodeRDD以及edgeRDD
\end{lstlisting}

其中的核心是找最大匹配边,这里采用了两种策略,一种是Heavy Edge,一种是Random Heavy Edge。
\begin{itemize}
    \item Heavy Edge算法每次通过reduce算子找到两个端点都没有匹配过的边里面最大权重的边。这里使用了filter和reduce算子
    \item Random Heavy Edge算法每次随机找一个点,找出与这个点连接的所有边的最大权重。这里使用了takeSample、filter与reduce算子
\end{itemize}

Heavy Edge的缺点是每次只找最大的匹配边,粗化过程需要多次调用这样的最大匹配算法,容易形成匹配边聚集,影响划分效果。
相比之下,Random Heavy Edge通过随机采样,避免匹配点聚集,让划分更加均衡,但是依赖随机采样的数据点。

\subsubsection{粗化过程中的数据结构更新}

实时更新\texttt{nodeRDD}以及\texttt{edgeRDD}的核心思路是通过filter和map算子的重复计算。

首先求出A和B的共同近邻节点,对于节点A和节点B都连接到的节点C,设置$edgeWeight=edgeWeight(AC)+edgeWeight(BC)$,此处使用了map算子。

同时新建一个\texttt{Node}类,设置其近邻为A与B的并集,组成节点列表composition由nodeA和nodeB组成,新节点的权重为原来的节点A和节点B的权重之和。
表示是否匹配的变量\texttt{isMark}设置为\texttt{true}。

在进行节点合并时,首先删去节点B,然后把节点A替换为新生成的节点。
这里通过map算子处理\texttt{nodeRDD},对于节点A,直接重新赋值为new Node。
根据new Node得到一个\texttt{edgeMap},是(起点,终点)到权重的映射,表示需要更新的边,用于更新\texttt{edgeRDD}。

接下来需要处理与节点A和节点B相连的其他节点。
\begin{itemize}
    \item 对于与A和B都连接的点D
        连接到new Node的\texttt{edgeWeight}更新为\texttt{edgeWeight(DA)+edgeWeight(DB)}
    \item 对于与只与A连接的点D
        连接到new Node的\texttt{edgeWeight}更新为$edgeWeight(DA)$
    \item 对于与只与B连接的点D
        连接到new Node的\texttt{edgeWeight}更新为$edgeWeight(DB)$
\end{itemize}

然后更新\texttt{edgeRDD},用map算子更新每一条记录,更新权重。
每条边只要有一个点被匹配,\texttt{isMark}参数就被设置为\texttt{true}。

合并两个节点的邻居的Scala代码如下:

\begin{lstlisting}[language=Scala]
unionMap = nodeA.neighbour++nodeB.neighbour
intersetNeighbour = nodeA.neighbour.keySet & nodeB.neighbour.keySet
unionMap.map(
   x=>if(x in intersetNeighbour)
    (x._1,nodeA.edgeWeight(x._1) + nodeB.edgeWeight(x._1))
)
\end{lstlisting}

node RDD在更新过程中会维护一个需要更新的边表(需要更新的边较少,内存可以存下来),更新的Scala代码如下:

\begin{lstlisting}[language=Scala]
    
val newNode = mergeNode(node1,node2,level)
var neighbourEdgeMap: Map[(String,String),Double] =
    newNode.getNeighbour.map(x=>((newNode.getIdx,x._1),x._2))

graph.nodeRDD = graph.nodeRDD.filter(
    _.getIdx!=node2.getIdx
).map(x=>
    if(x.getIdx==node1.getIdx) newNode
    else{
        val weight = x.edgeWeight(node1)+x.edgeWeight(node2)

        if(weight!=0) {
            neighbourEdgeMap += (x.getIdx,newNode.getIdx)->weight
            x.popNeighbour(node1).popNeighbour(node2).
                    pushNeighbour((newNode.getIdx,weight))
        }
        else x
    }
)
\end{lstlisting}

\texttt{edgeRDD}可以根据之前需要更新的边表来更新,Scala代码如下

\begin{lstlisting}[language=Scala]
graph.edgeRDD = graph.edgeRDD.filter
    { x =>
        !((x._1 == node1.getIdx && x._2 == node2.getIdx) ||
                (x._1 == node2.getIdx) && (x._2 == node1.getIdx))
    }

graph.edgeRDD=graph.edgeRDD.map(
    x=>
            //which node are node1,node2
        if(x._1==node1.getIdx||x._1==node2.getIdx)
            (newNode.getIdx,x._2,x._3,true)
        else if(x._2==node1.getIdx||x._2==node2.getIdx)
            (x._1,newNode.getIdx,x._3,true)
        else x
).map(
    x=>
        if(neighbourEdgeMap.contains((x._1,x._2)))
            (x._1,x._2,neighbourEdgeMap((x._1,x._2)),true)
        else x
).distinct()

\end{lstlisting}

另外,粗化过程的最大匹配每次都是组合两个点,需要记录这两个点的信息。
在\texttt{Node}类的属性上加入了\texttt{composition}和\texttt{compositionLevel}属性,分别表示这个节点由那两个节点组合而成,以及是在粗化的哪一个阶段组合而成。

\subsubsection{初步划分}

初步划分主要通过调用谱聚类完成。
前面提到Heavy Edge策略的时候,空间上容易形成聚集,而这些聚集的点与其他点的连接权重还会增加,所以这些点很容易聚成一类,解决方法是对拉普拉斯矩阵用点的weight进行归一化

谱聚类算法涉及到度矩阵的时候,使用度矩阵归一化,目标是求$D^{-1/2}LD^{-1/2}$的特征值和特征向量。
现在归一化节点权重也采用相似的思路,$L$矩阵每一个项变为:
$$L_{ij}/sqrt(D_i)sqrt(D_j)weight(i)weight(j)$$

实验显示均衡性的改善还不够,这里使用节点权重$\alpha$次方进行归一化:
$$L_{ij}/sqrt(D_i)sqrt(D_j)weight(i)^{\alpha} weight(j)^{\alpha}$$

对应的矩阵形式为:
$$W_{node}^{-\alpha}D^(-1/2)LD^(-1/2)W_{node}^{-\alpha}$$
实现过程中发现$\alpha=2$效果较好。

\subsubsection{细化过程}

细化过程本质上是递归的分解组合点,是粗化的逆过程,类似一个堆栈的出栈过程。
需要根据每个点记录的\texttt{compositionLevel}属性确定“出栈”的顺序,直到\texttt{compositionLevel}说明无需继续细化。
细化过程是针对\texttt{nodeRDD}展开的,用flatMap算子对当前细化层次的节点分解其\texttt{composition},得到两个新的点。
对于不需要分解的点直接拷贝过来。
通过合并两部分\texttt{nodeRDD}得到细化图的\texttt{nodeRDD},然后调用KL算法微调。
本项目实现的KL算法可以支持$K$路划分。
scala代码如下

\begin{lstlisting}[language=Scala]
var refinedGraph = graph
var refineLevel = level

//refine Level = 0 indicates that nodes all
while(refineLevel!=0){
    /** Step 1: Split coarsen node to refined nodes. */

    // 1.1 Find coarsen nodes and refine them (new node)
    var refinedNodeRDD = refinedGraph.nodeRDD.filter(x=>x.getComposLevel==refineLevel)

    refinedNodeRDD = refinedNodeRDD.map(_.setCompositionPartition())

    // 1.2 Save the nodes which don't need to refine
    val nodeRDD = refinedGraph.nodeRDD.filter(x=>x.getComposLevel!=refineLevel)
    refinedNodeRDD = refinedNodeRDD.flatMap(x=>x.getComposition)

    // 1.3 Union two parts of refined nodes.
    refinedGraph.nodeRDD = refinedNodeRDD.union(nodeRDD)


    /** Step 2: Partitioning */
    val assignment = refinedGraph.nodeRDD.map(x=>(x.getIdx,x.getPartition))

    refinedGraph = KernighanLin.partition(refinedGraph, assignment, needMaxGain = true)
    refinedGraph.nodeRDD = refinedGraph.nodeRDD.map(_.setChosen(false))

    refineLevel-=1 //refine Level decrease 1, up refine
}
\end{lstlisting}

% \section{测试数据}
\subsection{数据集简介}

采用两个无向图,一个有向图做实验

1、来自AAAI论文开源的数据The Network Data Repository with Interactive Graph Analytics and Visualizatio,  网址:http://networkrepository.com/aves-wildbird-network-3.php,是一个带权无向图,https://toreopsahl.com/datasets/#Cross_Parker

2、Zachary karate club network ,是一个无权无向图(默认为1)图领域的经典数据集,网址:http://konect.cc/networks/ucidata-zachary/

3、Intra-organisational networks,来自知名数据集网站https://toreopsahl.com/datasets/#Cross_Parker

评价指标:定义所有子图的内部连边之和为I,所有子图之间连边权重之和为E,则划分的质量为I/E

\subsection{Hash Partition}
Hash Partition没有考虑到子图之间的连接关系,只是简单的根据hash分区,因此效果较差,下面实验了分区数位2和3的情况

分区数 &划分performance
2 &0.9088310711082526
3 &0.5066518381951788

\subsection{Kernighan–Lin algorithm}

Kernighan–Lin algorithm通过随机划分数据集来做实验,因为Kernighan–Lin algorithm对初始化数据集划分比较敏感的影响,因此我们实验设置不同seed的来初始化划分数据集,Kernighan–Lin algorithm算法的图划分质量如下

seed  &划分performance 
324   &9.353000586441842
3241  &11.619001054402
1234  &25.071660045156847

另外,我们还画出了三种seed的情况下,Kernighan–Lin algorithm算法每次数据点交换迭代过程中,图划分质量的变化曲线,
可见,Kernighan–Lin algorithm算法的贪心策略还是有效的,每次交换数据都能带来performance的提高

\subsection{谱聚类}
谱聚类求归一化的拉普拉斯矩阵的特征值和特征向量的时候,是用幂迭代法实现的。因此需要设置最大迭代次数,下面实验了
最大迭代次数为20,10,20三种情况下,图划分的质量以及运行时间。
最大迭代次数 &划分performance      &运行时间
2           &39.32289045366845    &4.134
10          &70.58323798892366    &7.184
20           &70.58323798892366    &11.516
不难得出结论:在AAAI的图数据集上,谱聚类在10轮之内就收敛了。再增加迭代次数对提高图划分的质量没有任何帮助。此外,适当降低迭代次数,能过显著降低运行时间


\subsection{Metis algorithm}
\subsection{最大匹配策略实验}
在实际实验过程中发现,使用naive的heavy edge策略在进行多次粗化之后匹配点会形成聚集,使得划分阶段的谱聚类的划分不平衡。前面提到可以通过weight normalization来在一定程度上缓解这种不平衡划分,下面固定参数c=35,分别实验了heavy edge,heavy edge with WN(结合weight normalization),random heavy edge三种策略下,划分的质量比较:

策略        &划分performance   
heavy edge &53.95506559164985
heavy edge with WN &78.37563264477501
random heavy edge &79.52152205384091
random heavy edge &79.52152205384091

可以看出,1、由于heavy edge划分不平衡,因此performance较低;
2、weight normalization技术可以在一定程度上提高最终划分的performance
3、random heavy edge划分已经比较平衡,因此使用了weight normalization之后,对图划分的performance的提高不明显

\subsection{参数c选取实验}
c控制粗化的程度,当粗化得到点的个数小于等于c*k个的时候,粗化将会停止。一般来说c越大的话,粗化的程度越低,
信息损耗越少,划分的performance越;但是对partition阶段划分算法的性能要求越高,但是本系统瓶颈在粗化阶段,因此c越大,时间消耗越少。
c    &划分performance    &运行时间
c=15 &77.94451753134379 &120.644
c=25 &79.37368802905935 &114.546
c=35 &79.52152205384091 &80.608
c=50 &79.52152205384091 &73.314




\section{总结}

图划分问题普遍存在较高的计算复杂度,但图数据在现实中具有重要的价值,这也是人们不断追求更简易的图划分算法的原因。
上文提到的四种图划分算法在处理方法与结构上各具特色,也各有优劣。

传统图划分算法可以处理节点数和边数较少的图,包括局部改进图划分算法和全局图划分算法,其中局部改进图划分算法中比较经典的是KL算法,全局图划分算法中比较经典的是谱划分法和多层图划分METIS算法。
然而这些算法具有较高的时问复杂度,无法处理节点数为百万级以上的图。
所以接下来我们可以对分布式图算法进行进一步的研究,从而处理大数据时代中大规模的图划分问题。

\end{document}
