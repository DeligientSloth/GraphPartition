\section{测试数据}
\subsection{数据集简介}

采用两个无向图,一个有向图做实验

1、来自AAAI论文开源的数据The Network Data Repository with Interactive Graph Analytics and Visualizatio,  网址:http://networkrepository.com/aves-wildbird-network-3.php,是一个带权无向图,https://toreopsahl.com/datasets/#Cross_Parker

2、Zachary karate club network ,是一个无权无向图(默认为1)图领域的经典数据集,网址:http://konect.cc/networks/ucidata-zachary/

3、Intra-organisational networks,是一个有向无权图,来自知名数据集网站https://toreopsahl.com/datasets/#Cross_Parker

评价指标:定义所有子图的内部连边之和为I,所有子图之间连边权重之和为E,则划分的质量为I/E

\subsection{Hash Partition}
Hash Partition没有考虑到子图之间的连接关系,只是简单的根据hash分区,因此效果较差,下面实验了分区数位2和3的情况

分区数 &划分performance
2 &0.9088310711082526
3 &0.5066518381951788

\subsection{Kernighan–Lin algorithm}

Kernighan–Lin algorithm通过随机划分数据集来做实验,因为Kernighan–Lin algorithm对初始化数据集划分比较敏感的影响,因此我们实验设置不同seed的来初始化划分数据集,Kernighan–Lin algorithm算法的图划分质量如下

不同初始化划分下,Kernighan–Lin algorithm的performance的差异
seed  &划分performance 
324   &9.353000586441842
3241  &11.619001054402
1234  &25.071660045156847

另外,我们还画出了三种seed的情况下,Kernighan–Lin algorithm算法每次数据点交换迭代过程中,图划分质量的变化曲线,
可见,Kernighan–Lin algorithm算法的贪心策略还是有效的,每次交换数据都能带来performance的提高

Kernighan–Lin algorithm随点交换的performance变化图
\figure{../figure/KL.png}

\subsection{谱聚类}
谱聚类求归一化的拉普拉斯矩阵的特征值和特征向量的时候,是用幂迭代法实现的。因此需要设置最大迭代次数,下面在AAAI network下面实验了最大迭代次数为20,10,20三种情况下,图划分的质量以及运行时间。


      AAAI network下谱聚类的performance与运行时间与最大迭代次数的关系
最大迭代次数 &划分performance      &运行时间
2           &39.32289045366845    &4.134
10          &70.58323798892366    &7.184
20           &70.58323798892366    &11.516

不难得出结论:在AAAI的图数据集上,谱聚类在10轮之内就收敛了。再增加迭代次数对提高图划分的质量没有任何帮助。此外,适当降低迭代次数,能过显著降低运行时间
但是在Zachary数据集上,迭代次数有所不一样

    Zachary network下谱聚类的performance与运行时间与最大迭代次数的关系
最大迭代次数 &划分performance      &运行时间
20          &4.2                  &7.573
30          &5.5                   &9.28
40          &5.5                  &13.047
可见,在Zachary network上面,需要20轮以上才会收敛,因此最大迭代次数需要根据数据集做一些试探性的实验才能确定。

\subsection{Metis algorithm}
\subsection{最大匹配策略实验}
在实际实验过程中发现,使用naive的heavy edge策略在进行多次粗化之后匹配点会形成聚集,使得划分阶段的谱聚类的划分不平衡。前面提到可以通过weight normalization来在一定程度上缓解这种不平衡划分,下面固定参数c=35,分别实验了heavy edge(HE),heavy edge with WN(结合weight normalization,HEWN),random heavy edge(RHE),random heavy edge with weight normalization(RHEWN)四种策略下,划分的质量比较:

     AAAI network下metis四种最大匹配策略的performance
策略        &划分performance   
heavy edge(HE) &53.95506559164985
heavy edge with WN(HEWN) &78.37563264477501
random heavy edge(RHE) &79.52152205384091
random heavy edge with weight normalization(RHEWN) &79.52152205384091

可以看出,1、由于heavy edge划分不平衡,因此performance较低;
2、weight normalization技术可以在一定程度上提高最终划分的performance
3、random heavy edge划分已经比较平衡,因此使用了weight normalization之后,对图划分的performance的提高不明显

\subsection{参数c选取实验}
c控制粗化的程度,当粗化得到点的个数小于等于c*k个的时候,粗化将会停止。一般来说c越大的话,粗化的程度越低,
信息损耗越少,划分的performance越;但是对partition阶段划分算法的性能要求越高,但是本系统瓶颈在粗化阶段,因此c越大,时间消耗越少。

AAAI network下metis的performance与运行时间与参数c的关系
c    &划分performance    &运行时间
c=15 &77.94451753134379 &120.644
c=25 &79.37368802905935 &114.546
c=35 &79.52152205384091 &80.608    
c=50 &79.52152205384091 &73.314


\subsection{谱聚类与metis算法的横向对比}
下面分别比较当划分子图个数k为2个,3个的时候,谱聚类与metis在三个数据集上面得到的图划分质量
在AAAI network上的谱聚类的最大迭代次数取10,metis算法的c取35,
在Zachary network上谱聚类的最大迭代次数取30,metis算法的c取15
在Intra-organisational networks上谱聚类最大迭代次数取20,metis算法的c取20


        k=2时,三种数据集下谱聚类与metis的performance与运行时间的横向比较
                                 k=2
                                  AAAI network                       Zachary network(30)     
algorithm      &peformance           &运行时间   &peformance  &运行时间
谱聚类          &70.58323798892366    &7.184      &5.5      &9.28
metis           &79.52152205384091    &80.608     &15.75    &16.455

Intra-organisational networks
谱聚类   &4.0835509138381205  &7.402
metis   &4.607476635514018 &22.008

k=3时,AAAI数据集下谱聚类与metis的performance与运行时间的横向比较
                                 AAAI network                          
algorithm                   &peformance           &运行时间   
谱聚类             &25.157246974213972           &16.002      
metis             &34.0869985482627             &142.937  

可以看出,在划分子图个数为2个和3个的时候,metis在不同数据集上,划分的performance都要比谱聚类好,但是metis的运行时间要显著高于谱聚类算法。
