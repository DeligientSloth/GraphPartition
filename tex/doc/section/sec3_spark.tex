\section{基于Spark的图划分}

Apache Spark是一个开源集群运算框架,最初是由加州大学柏克莱分校AMPLab所开发。
相对于Hadoop的MapReduce会在运行完工作后将中介数据存放到磁盘中,Spark使用了存储器内运算技术,能在数据尚未写入硬盘时即在存储器内分析运算。
Spark在存储器内运行程序的运算速度能做到比Hadoop MapReduce的运算速度快上100倍,即便是运行程序于硬盘时,Spark也能快上10倍速度。

Apache Spark项目包含下列几项:弹性分布式数据集(RDDs)、Spark SQL、Spark Streaming、MLlib和GraphX。
Spark提供了分布式任务调度,调度和基本的I/O功能。
Spark的基础程序抽象是弹性分布式数据集(RDDs),RDD一个可以并行操作、有容错机制的数据集合。
RDDs可以透过引用外部存储系统的数据集创建(例如:共享文件系统、HDFS、HBase或其他 Hadoop 数据格式的数据源),
或者是透过在现有RDDs的转换而创建(比如:map、filter、reduce、join等等)。 

\subsection{图数据实现}

为了在Spark上进行图划分,本文将先介绍\texttt{Node}和\texttt{Graph}这两个类的基本情况。

\subsubsection{\texttt{Node}类}

\texttt{Node}类是图数据处理的基础。

\begin{table}[htbp]
    \centering
    \caption{\texttt{Node}类主要属性}
    \begin{tabular}{ccc}
        \hline
        属性& 类型 & 定义\\
        \hline
        idx         & String             & 节点的唯一Id \\
        neighbour   & Map[String,Double] & 节点的所有近邻节点\\
        E           & Double             & 外部权重,即节点与其他子图内的节点的连接权重和\\
        I           & Double             & 内部权重,即节点与本子图内的节点的连接权重和\\
        partition   & Int                & 节点所在子图的Id\\
        chosen      & Boolean            & 节点是否与其他子图的节点交换过\\
        composition & List[Node]         & 该节点的组成节点列表(仅用于节点聚合/拆分过程)\\
        composLevel & Int                & 该节点的聚合程度(仅用于节点聚合/拆分过程)\\
        isMark      & Boolean            & 节点是否与其他节点匹配过 \\
        weight      & Double             & 节点的权重\\
        \hline
        \centering
    \end{tabular}
\end{table}

\subsubsection{\texttt{Graph}类}

\texttt{Graph}类是将节点和边组合起来。

\begin{table}[htbp]
    \centering
    \caption{\texttt{Graph}类主要属性}
    \begin{tabular}{ccc}
        \hline
        属性& 类型 & 定义\\
        \hline
        nodeNum & Long & 图内节点的个数 \\
        edgeRDD & RDD[(Str, Str, Double, Bool)] & 图内所有边数据的RDD形式\\
        nodeRDD & RDD[Node] & 图内所有节点的RDD形式\\
        \hline
        \centering
    \end{tabular}
\end{table}

\texttt{edgeRDD}的每项由以下四个部分组成:起点节点Id值、终点节点Id值、边权重、是否已经被匹配。

\subsection{哈希划分算法实现}

\begin{lstlisting}[language=Scala]
import org.apache.spark.{HashPartitioner, TaskContext}
import util.Graph

object HashGraphPartition {
    def partition(graph: Graph, partitions: Int): Graph = {
        val assigenment = graph.nodeRDD.map(x => (x.getIdx, 0)).partitionBy(
            new HashPartitioner(partitions)).map(x => (x._1, TaskContext.getPartitionId))
        graph.buildPartitionGraph(assigenment)
    }
}
\end{lstlisting}

\subsection{谱聚类算法实现}

在Spark中实现谱聚类算法时,可以调用MLlib类库中的Power iteration clustering算法。

幂迭代聚类(Power iteration clustering)由Frank Lin和William W.Cohen提出,最早发表于ICML 2010。
在数据归一化的逐对相似矩阵上,使用截断的幂迭代,PIC寻找数据集的一个超低维嵌入(低纬空间投影,embedding )。
这种嵌入恰好是很有效的聚类指标,使它在真实数据集上总是好于广泛使用的谱聚类方法(比如说NCut)。

\begin{algorithm}[htbp]
\caption{PIC算法流程}
\SetAlgoLined
\KwIn{按行归一化的关联矩阵$W$}
\KwIn{期望聚类数$k$}
随机选取一个非零初始向量$v^0$ \\
\Repeat{$\left|\delta^{t}-\delta^{t-1}\right| \simeq 0$}{
    $ v^{(t+1)}=\frac{W v^{(t)}}{\left|W v^{(t)}\right|_{1}} $\\
    $\delta^{(t+1)}=\left|v^{(t+1)}-v^{(t)}\right|$
    增加$t$值
}
使用k-means算法对向量$v^t$中的点进行聚类
\KwOut{类$C_1,C_2,\cdots,C_K$}
\end{algorithm}

\begin{lstlisting}[language=Scala]
def partition(graph: Graph, partitions: Int, maxIter: Int, weightNorm:Boolean): Graph = {

    var weight:scala.collection.Map[String,Double]=null
    if(weightNorm==true)
        weight = graph.nodeRDD.map(x=>(x.getIdx,x.getWeight)).collectAsMap()

    def weight_norm(sim:Double,x:String,y:String):Double={
        if(weightNorm) sim/math.pow(weight(x)*weight(y), 2)
        else sim
    }

    val simEdge = graph.edgeRDD.map(
        x => (x._1.toString.toLong,
                x._2.toString.toLong,
                gaussianSimilarity(weight_norm(x._3,x._1,x._2), 1)))

    val modelPIC = new PowerIterationClustering()
            .setK(partitions) // k : 期望聚类数
            .setMaxIterations(maxIter) //幂迭代最大次数
            .setInitializationMode("degree") //使用度初始化
            .run(simEdge)

    val assign = modelPIC.assignments.map(x => (x.id.toString, x.cluster))
    graph.buildPartitionGraph(assign)
    graph
}
\end{lstlisting}

\begin{lstlisting}[language=Scala]
def gaussianSimilarity(dist: Double, sigma: Double): Double =
    math.exp(-dist / (2 * sigma * sigma))
\end{lstlisting}


\subsection{Kernighan-Lin算法实现}

\begin{lstlisting}[language=Scala]
\end{lstlisting}

\subsection{Metis算法实现}

\begin{lstlisting}[language=Scala]
def partition(graph: Graph, c:Int, mode:String,weightNorm:Boolean): Graph={
    var partitionedGraph = graph
    var level:Int=0 

    // 1.coarsening phase
    val coarsenRes = coarsen(graph, c,mode)
    partitionedGraph = coarsenRes._1 // coarsen Graph
    level = coarsenRes._2  //coarsen level

    // 2.partitioning phase
    partitionedGraph = initialPartition(partitionedGraph, weightNorm:Boolean)
    partitionedGraph.printGraph()

    // 3.un-coarsening phase
    partitionedGraph = refine(partitionedGraph,level) //need level to decide refine order

    partitionedGraph
}
\end{lstlisting}

\begin{lstlisting}[language=Scala]
private def coarsen(graph: Graph, c: Int,mode:String): (Graph,Int)={
    /** Step 1: Coarsening Phase (via Maximal Matching Algorithm)
        * input:  origin graph G_o,
        *         coarsening parameter c,
        *         mode
        * output: coarsen graph G_c
    * */
    var coarsenedGraph = graph
    var level = 0
    while(coarsenedGraph.nodeNum > c*k){
        level += 1
        coarsenedGraph = maximalMatching(coarsenedGraph, mode,level)
        coarsenedGraph = refreshMarkedFlag(coarsenedGraph)
    }
    (coarsenedGraph,level)
}
\end{lstlisting}

\begin{lstlisting}[language=Scala]
private def initialPartition(graph: Graph, useWeightNorm:Boolean): Graph={
    /** Step 2: Partitioning Phase (via Spectral Clustering)
        * input:  graph - coarsened graph g
        *         useWeightNorm - whether to normalize the weights.
        * output: splitGraph - initial partition graph
        * */
    val splitGraph = SpectralClustering.partition(graph, k, 40, useWeightNorm)
    splitGraph
}
\end{lstlisting}


\begin{lstlisting}[language=Scala]
private def refine(graph: Graph,level:Int): Graph={
    /** Step 3: Refining Phase (via Kernighan-Lin Algorithm)
        * input: graph - initial partition graph
        *        level - refine level
        * output: refined graph
        * */
    var refinedGraph = graph
    var refineLevel = level
    while(refineLevel!=0){
        /** Step 1: Split coarsen node to refined nodes. */

        // 1.1 Find coarsen nodes and refine them (new node)
        var refinedNodeRDD = refinedGraph.nodeRDD.
                            filter(x=>x.getComposLevel==refineLevel)
        refinedNodeRDD = refinedNodeRDD.map(_.setCompositionPartition())

        // 1.2 Save the nodes which don't need to refine
        val nodeRDD = refinedGraph.nodeRDD.filter(x=>x.getComposLevel!=refineLevel)
        refinedNodeRDD = refinedNodeRDD.flatMap(x=>x.getComposition)

        // 1.3 Union two parts of refined nodes.
        refinedGraph.nodeRDD = refinedNodeRDD.union(nodeRDD)

        /** Step 2: Partitioning */
        val assignment = refinedGraph.nodeRDD.map(x=>(x.getIdx,x.getPartition))
        refinedGraph = KernighanLin.partition(refinedGraph, assignment, needMaxGain = true)
        refinedGraph.nodeRDD = refinedGraph.nodeRDD.map(_.setChosen(false))
        refineLevel-=1 //refine Level decrease 1, up refine
    }

    refinedGraph.nodeNum = refinedGraph.nodeRDD.count()
    refinedGraph
}
\end{lstlisting}

